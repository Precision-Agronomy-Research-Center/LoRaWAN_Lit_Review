% lorawan_review_recruitment_slides.tex
% Build: pdflatex lorawan_review_recruitment_slides.tex
% Notes: If you want speaker notes, compile with:
%   pdflatex "\def\shownotes{1}\input{lorawan_review_recruitment_slides.tex}"

\documentclass[aspectratio=169,11pt]{beamer}

% -----------------------
% Optional speaker notes
% -----------------------
\usepackage{pgfpages}
\newif\ifshownotes
\ifdefined\shownotes
  \shownotestrue
\else
  \shownotesfalse
\fi

\ifshownotes
  \setbeameroption{show notes on second screen=right}
\else
  \setbeameroption{hide notes}
\fi

% -----------------------
% Minimal theme styling
% -----------------------
\usetheme{default}
\usecolortheme{default}
\setbeamertemplate{navigation symbols}{}
\setbeamertemplate{footline}[frame number]

\usepackage[T1]{fontenc}
\usepackage[utf8]{inputenc}
\usepackage{lmodern}
\usepackage{hyperref}
\hypersetup{hidelinks}

% -----------------------
% Metadata
% -----------------------
\title{How Is LoRaWAN Actually Used in Agriculture?}
\subtitle{A living systematic review to connect deployments to decision support}
\author{Avinash Baskaran}
\institute{Collaborative review project (GitHub-based workflow)}
\date{} % Leave blank

% -----------------------
% Convenience macros
% -----------------------
\newcommand{\tightlist}{\setlength{\itemsep}{4pt}\setlength{\parskip}{0pt}}

\begin{document}

% -----------------------
% Title slide
% -----------------------
\begin{frame}
  \titlepage
  \note{
  Hi everyone. In the next 20 minutes I want to recruit collaborators for a living systematic review.
  The question is simple: how is LoRaWAN actually used in agricultural decision support, and when does it work in practice.
  You do not need to be a LoRaWAN specialist to contribute.
  }
\end{frame}

% -----------------------
% Slide: Why now
% -----------------------
\begin{frame}{Why this matters now}
\begin{itemize}\tightlist
  \item LoRaWAN is widely proposed as the backbone for agricultural sensing
  \item Papers often report network metrics without showing decision impact
  \item Practitioners want to know: will this help me decide and act?
\end{itemize}
\note{
LoRaWAN is everywhere in agriculture sensing proposals, but adoption does not equal understanding.
A lot of work reports range, packets, RSSI, and battery life, but does not connect those to decisions like irrigation timing,
disease response, livestock movement, or maintenance scheduling.
The goal here is to make the evidence legible for system designers and practitioners.
}
\end{frame}

% -----------------------
% Slide: The gap
% -----------------------
\begin{frame}{The gap in the literature}
\begin{itemize}\tightlist
  \item Deployments are heterogeneous: scale, sensors, gateway layouts, backhaul
  \item Performance is reported inconsistently across studies
  \item Decision support is often asserted, not evaluated
\end{itemize}
\note{
When you try to compare studies, you hit fragmentation fast.
Some are simulations, some are single-farm pilots, some are multi-site networks.
Metrics differ, reporting differs, and decision linkage is often missing.
That makes it hard to answer practical questions like when LoRaWAN is the right choice.
}
\end{frame}

% -----------------------
% Slide: Existing reviews
% -----------------------
\begin{frame}{Why existing reviews do not resolve it}
\begin{itemize}\tightlist
  \item Many reviews organize by protocol layer or feature lists
  \item Useful for capabilities, less useful for decision relevance
  \item Missing: recurring failure modes and operational constraints
\end{itemize}
\note{
Existing LPWAN or LoRaWAN reviews are valuable, but they often answer a different question.
They summarize capabilities rather than connecting deployments to decision outcomes.
We want to bring in operational realities: maintenance burden, gateway placement constraints,
seasonality, interference, and what breaks over time.
}
\end{frame}

% -----------------------
% Slide: Core question
% -----------------------
\begin{frame}{Core research question}
\begin{block}{Primary question}
How is LoRaWAN used in agricultural decision support systems, and what factors determine whether it effectively supports decisions in practice?
\end{block}

\begin{itemize}\tightlist
  \item What decisions are being supported, if any?
  \item What constraints repeatedly limit usefulness?
  \item When do alternatives outperform (cellular, NB-IoT, mesh)?
\end{itemize}
\note{
This is a decision-centered review. Not a protocol benchmark paper.
We want to understand what decisions are actually supported, where systems fail, and where alternatives make more sense.
}
\end{frame}

% -----------------------
% Slide: What we will produce
% -----------------------
\begin{frame}{What the review will produce}
\begin{itemize}\tightlist
  \item A reproducible corpus of LoRaWAN-in-agriculture deployments
  \item A standardized extraction table of system details and reported outcomes
  \item A synthesis of recurring patterns, constraints, and gaps
  \item A living artifact that updates as new work appears
\end{itemize}
\note{
We are building a shared dataset first, then writing follows.
The output is not just a manuscript. It is also the extraction table and the evidence map that others can reuse.
}
\end{frame}

% -----------------------
% Slide: Workflow
% -----------------------
\begin{frame}{Workflow (simple and auditable)}
\begin{enumerate}\tightlist
  \item Define protocol: research question, sources, inclusion criteria
  \item Run searches and add candidate papers
  \item Screen titles and abstracts, then full text
  \item Extract standardized fields into a shared table
  \item Synthesize findings and draft sections
  \item Periodically update with new literature
\end{enumerate}
\note{
This is a standard systematic review loop.
The difference is the collaboration and transparency: issues and pull requests, shared extraction, and regular updates.
}
\end{frame}

% -----------------------
% Slide: Inclusion criteria
% -----------------------
\begin{frame}{What we include (and exclude)}
\begin{columns}[T,onlytextwidth]
\column{0.5\textwidth}
\textbf{Include}
\begin{itemize}\tightlist
  \item LoRa or LoRaWAN used in an agricultural context
  \item Describes a system, deployment, or evaluation
  \item Reports at least some technical or operational details
\end{itemize}

\column{0.5\textwidth}
\textbf{Exclude}
\begin{itemize}\tightlist
  \item Purely theoretical work with no ag application
  \item Agriculture mentioned but LoRaWAN not actually used
  \item Insufficient detail to extract basic system information
\end{itemize}
\end{columns}
\note{
We want evidence that can be compared across studies.
If something is a conceptual proposal with no system details, it is usually out.
Grey literature can be in if it provides deployment detail that peer reviewed work misses.
}
\end{frame}

% -----------------------
% Slide: Data extraction
% -----------------------
\begin{frame}{What we extract from each study}
\begin{itemize}\tightlist
  \item Context: crop, livestock, environment, geography, scale
  \item System: sensors, sampling cadence, node power strategy
  \item Network: gateways, backhaul, topology assumptions
  \item Performance: reliability, range, latency proxies, battery life
  \item Decision link: what action is supported, how often, by whom
  \item Limitations: failures, constraints, maintenance burden
\end{itemize}
\note{
This is the practical core.
We will not require every field for every paper, but we will standardize what we try to capture.
Decision link is important: what action is claimed, and is it evidenced or just asserted.
}
\end{frame}

% -----------------------
% Slide: Contribution paths
% -----------------------
\begin{frame}{How you can contribute (multiple paths)}
\begin{itemize}\tightlist
  \item \textbf{Add papers:} suggest studies or deployments we missed
  \item \textbf{Screening:} help apply inclusion and exclusion criteria
  \item \textbf{Extraction:} fill standardized fields for assigned papers
  \item \textbf{Synthesis:} summarize clusters and draft text later
  \item \textbf{Practice input:} share real deployment constraints and failures
\end{itemize}
\note{
I want to emphasize: you do not need to write prose to contribute.
Extraction and screening are valuable and are tracked for credit.
Practitioner input is also welcome, especially failure modes that do not appear in papers.
}
\end{frame}

% -----------------------
% Slide: What you get
% -----------------------
\begin{frame}{What contributors get}
\begin{itemize}\tightlist
  \item Co-authorship and contributor credit aligned with community standards
  \item A citable resource and a reusable extraction dataset
  \item Clearer evidence for design choices and technology selection
  \item A shared map of open problems and future research directions
\end{itemize}
\note{
This is intended to be high utility for both academic and applied people.
You gain a citable systematic review and a dataset you can reference in future work.
}
\end{frame}

% -----------------------
% Slide: Ask and next steps
% -----------------------
\begin{frame}{The ask (today)}
\begin{itemize}\tightlist
  \item If you have relevant papers, send them or open an issue
  \item If you can screen or extract 3 to 5 papers, join the first sprint
  \item If you have deployment experience, share constraints and failures
\end{itemize}

\vspace{6pt}
\begin{block}{Contact}
Email: \href{mailto:avinash@huminlabs.com}{avinash@huminlabs.com}
\end{block}

\note{
Close with a concrete request: send papers, volunteer for a small extraction sprint, or share practical experience.
Mention that you will provide a simple extraction template and assign small chunks to keep it lightweight.
}
\end{frame}

% -----------------------
% Slide: Q&A
% -----------------------
\begin{frame}{Questions}
\begin{center}
\vspace{12pt}
{\Large Q\&A}
\vspace{8pt}

\small
If helpful, I can share the protocol, search strings, and extraction template.
\end{center}
\note{
Invite questions. If the audience wants detail, offer the protocol and extraction template as the concrete artifact.
}
\end{frame}

\end{document}
